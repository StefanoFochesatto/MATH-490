\documentclass{beamer}
\usetheme{Boadilla}
\setbeamertemplate{caption}{\raggedright\insertcaption\par}
%typical packages
\usepackage{tikz,amsfonts,amsmath,pgfplots,float}

%typical tikz stuff
\tikzstyle{vertex}=[circle, draw, inner sep=0pt, minimum size=6pt,fill=white]
\newcommand{\vertex}{\node[vertex]}


\title{Automorphism Groups:}
\subtitle{Theorems 2.10 and 2.11}
\author{Stefano Fochesatto}
\institute{University of Alaska Fairbanks}
\date{\today}

\begin{document}

 %TITLE PAGE
 \begin{frame}
\titlepage
\end{frame}

%Second Page
\begin{frame}
\frametitle{Review}
\begin{center}
\begin{itemize}
	\item Automorphism: An isomorphism from a graph $G$ to itself. 
	\begin{itemize}
		\item A permutation of V(G) that preserves adjacency and non-adjacency
			\begin{figure}[H]
			\caption{f:(BC)}
			\centering
			\includegraphics[width=.45\textwidth]{"automapping".png}
			\end{figure}

	\end{itemize}
		\vfill
	\item Automorphism Group: The set of all automorphisms on $G$, form a group (denoted $Aut(G)$) under function composition. 
\end{itemize}
\end{center}
\end{frame}




%Third Page
\begin{frame}
\frametitle{Theorem 2.10}
\begin{center}
$\bold{Theorem \;2.10:}$ For every graph $G$, $Aut(G) \cong Aut(\overline{G})$
\begin{itemize}
	\item How do we prove this?
\end{itemize}
\end{center}
\end{frame}

%Fourth Page
\begin{frame}
\frametitle{Theorem 2.10}
\begin{center}

\begin{itemize}
\item $\bold{Direct:}$ WTS: For every $i\in Aut(G)$, that $i\in Aut(\overline{G})$.
	\begin{itemize}
		\item Consider an automorphism $i \in Aut(G)$.
		\vfill
		\item By definition $i: V(G) \to V(G)$ that preserves adjacency and non-adjacency. 
		\vfill
		\item We can apply $i$ to the set $V(\overline{G})$ since $V(G) = V(\overline{G})$.		
		\vfill
		\item Note a function that preserves adjacency in $G$ will preserve non-adjacency in $\overline{G}$. 
		\vfill
		\item Similarly a function that preserves non-adjacency in $G$ will preserve adjacency in $\overline{G}$. 
		\vfill
		\item Therefore by definition $i$ is an automorphism for $\overline{G}$.
		\vfill
		\item Thus $i \in Aut(\overline{G})$.
	\end{itemize}
\end{itemize}
\end{center}
\end{frame}



%Fifth Page
\begin{frame}
\frametitle{Theorem 2.10 Example:}
\begin{center}
		\begin{figure}[H]
		\caption{$i: (AB)(CD)$}
		\centering
		\includegraphics[width=.45\textwidth]{"210example".png}
		\end{figure}
		
		\begin{figure}[H]
		\caption{$i: (ABCD) \to (BADC)$}
		\centering
		\includegraphics[width=.45\textwidth]{"210examplecomp".png}
		\end{figure}


\end{center}
\end{frame}


% Sixth  Page
\begin{frame}
\frametitle{Theorem 2.11}
\begin{center}
$\bold{Theorem \;2.11:}$ The order of the automorphism group of a graph $G$ with order $n$ is a divisor of $n!$ and equals $n!$ if and only if $G = K_n$ or $G = \overline{K}_n$
	\begin{itemize}
		\item What does that even mean?
		\vfill
			\begin{itemize}
				\item if $|V(G)| = n$ then $|Aut(G)| \bigl\vert n!$
				\vfill
				\item If $G = K_n$ or $G = \overline{K}_n$ then $|Aut(G)| = n!$
				\vfill
			\end{itemize}
		\end{itemize}
\end{center}
\end{frame}

%Seventh Page
\begin{frame}
\frametitle{Theorem 2.11}
Recall
\begin{center}
\begin{itemize}
	\item $\bold{Symmetric\;Group:}$ The symmetric group $S_n$ is the group of all permutations on $n$ elements. Thus $|S_n| = n!$
	\begin{equation*}
		S_3 = 	
		\begin{pmatrix}
		(1)(2)(3) &(1)(23)   \\
		(123)& (2)(13)   \\
		(132)& (3)(12)
		\end{pmatrix}
	\end{equation*}
	\item $\bold{Lagrange's\; Theorem:}$ If $H$ is a subgroup of $G$, then $|G| = n|H|$ for some $n \in \mathbb{Z}$.
		\begin{itemize}
			\item This implies $H \bigl\vert G$.
		\end{itemize}
	\end{itemize}
\end{center}
\end{frame}

%Eighth Page
\begin{frame}
\frametitle{Theorem 2.11}
\begin{center}
\begin{itemize}
	\item $\bold{Direct:}$ WTS: if $|V(G)| = n$ then $|Aut(G)| \bigl\vert n!$
		\begin{itemize}
			\item Suppose a graph $G$ such that $|V(G)| = n$.
			\vfill
			\item By definition the of an Automorphism Group (permutation) we know that $Aut(G)$ is a group of permutations on $n$ elements that preserves (non)-adjacency,
			\vfill
			\item Note that $S_n$ is the group of $\bold{all}$ permutation on a set of $n$ elements. 
			\vfill
			\item Thus $Aut(G) \triangleright S_n$.
			\vfill
			\item By Lagrange's Theorem we know that $|Aut(G)| \bigl\vert |S_n|$, and by substitution we get $|Aut(G)| \bigl\vert n!$. 
		\end{itemize}
\end{itemize}
\end{center}
\end{frame}


%Ninth Page
\begin{frame}
\frametitle{Theorem 2.11}
\begin{center}
\begin{itemize}
	\item $\bold{Direct:}$ WTS: If $G = K_n$ or $G = \overline{K}_n$ then $|Aut(G)| = n!$
		\begin{itemize}
		\item Suppose $K_n$,
		\vfill
		\item By definition the of an Automorphism Group (permutation) we know that $Aut(K_n)$ is a group of permutations on $n$ elements that preserves (non)-adjacency.
		\vfill
		\item Since every vertex in $K_n$ is adjacent with the rest of the vertices, $\bold{every}$ permutation of $V(K_n)$ preserves (non)-adjacency.
		\vfill
		\item Therefore $Aut(K_n) \cong S_n$
		\vfill
		\item By Theorem 2.10 $Aut(K_n) \cong  Aut(\overline{K}_n)$
		\vfill
		\item Thus $|Aut(G)| = |Aut(\overline{K}_n| = n!.$
		\end{itemize}
\end{itemize}
\end{center}
\end{frame}






%TenthPage
\begin{frame}
\frametitle{Theorem 2.11 Example:}
\begin{center}
		\begin{figure}[H]
		\caption{Graph G:}
		\centering
		\includegraphics[width=.25\textwidth]{"211example".png}
		\end{figure}
\begin{itemize}
\item\begin{equation*}
	Aut(G) = \{(a)(b)(c), (a)(bc)\}
	\end{equation*}
\vfill
\item \begin{equation*}
	2 \bigl\vert 3!
	\end{equation*}
\end{itemize}
\end{center}
\end{frame}

%  Twnth Page
\begin{frame}
\frametitle{Theorem 2.11 Example:}
\begin{center}
		\begin{figure}[H]
		\caption{Graph G:}
		\centering
		\includegraphics[width=.75\textwidth]{"example2".png}
		\end{figure}
\end{center}
\end{frame}


\end{document}

%\end{document}
